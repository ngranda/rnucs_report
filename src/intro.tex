\section{Introduction}\label{sc:intro}
During operation of accelerator based systems, like the Spallation Neutron
Source (SNS), the nuclear reactions happening in the system produce high
energy neutrons that penetrate deep into the components of the system.
The materials of the components become activated from the neutron irradiation
and release high energy photons that persists even after shutdown. These
photons can negatively impact the health of a human body. During maintenance,
personnel come in close contact with these components, therefore it becomes
important to quantify the dose rates from these photons.
Quantification of these dose rates in nuclear systems have traditionally been
performed computationally and workflows exist to aid the calculation.
A common way to solve these problems is by coupling the neutron transport and
photon transport calculations using activation analysis, known as the Rigorous
2 Step (R2S) method. The R2S method has been implemented at UW as an automated
workflow with fidelity determined by a superimposed mesh, and has been added
to the Python for Nuclear Engineers  (PyNE) package. The UW-R2S capabilities
only extend to energies in the fusion system region (20 MeV) and future work
need to be performed to extend the capabilities to accelerator systems.
In higher energy regions such as those encountered in accelerator-based
systems, R2S workflows have been developed to be able to perform a complete
analysis in these systems. One such workflow relies on a radionuclide
inventory tally (RNUCS) implemented in MCNPX to obtain reaction rates across
the full energy spectrum.  These results are used during the activation step,
producing photon sources, but with fidelity limited to individual geometric
volumes.
\newpage
