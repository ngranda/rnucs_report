\section{Analysis}\label{sc:analysis}
In order to asses the correct progress of the rnucs-mesh workflow, a few 
analysis were performed. 

\subsection{Geometries}
Two toy problems were identified to asses the correctness of the 
rnuc-mesh workflow. 
Toy problem 1 consists of a mercury box with geometrical especification as 
listed in Table \ref{tab:geom_dim} 
Toy problem 2 consists of a mercury box surrounded by a steel box. The 
specifications can be found in Table \ref{tab:geom_dim}

% ------------- Geometry specification table ------------------------
\begin{table}[ht]
\begin{tabular}{|c|c|c|c|}
  \hline
  Geometry       & X position (cm) & Y position (cm) & Z position (cm) \\
  \hline  \hline
  Toy problem 1  & -11, 11         & -22, 22         & -93, 93         \\
  \hline
  Toy problem 2  & -22, 22         & -44, 44         & -186, 186       \\
  \hline
\end{tabular}
\caption{Toy problems dimensions}
\label{tab:geom_dim}
\end{table}


Six geometries were created, three for each toy problem. 
The first geometry was built as  described above, the second geometry 
was split into 8 equal cells and the last one was split 64 equal cells. 

Spliting toy problem 2 into 8 equal cells required that the materials 
be mixed in a 3:1 ratio


\subsection{Monte Carlo Run}
All problems were run with a 1 GeV proton source in the z direction and 
with 1E6 particles. 
The following runs were performed:

\begin{list}
- rnucs on cells 
- 1x1x1 mesh
- 2x2x2 mesh
- 4x4x4 mesh 
- 2x2x2 split
- 4x4x4 split
\end{list}



