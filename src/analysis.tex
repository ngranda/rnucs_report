\section{Analysis}\label{sc:analysis}
In order to asses the correct progress of the rnucs-mesh workflow, a few 
analysis were performed on two toy problems. 

\subsection{Problem Description}
Two toy problems were identified to asses the correctness of the 
rnuc-mesh workflow. 
Toy problem 1 (Figure \ref{tp1}) consists of a mercury box with geometrical specifications as 
listed in Table \ref{tab:geom_dim}. 
Toy problem 2 (Figure \ref{tp2}) consists of a mercury box surrounded by a steel box. The 
specifications can be found in Table \ref{tab:geom_dim}


% ------------- Geometry specification table ------------------------
\begin{table}[ht]
\begin{centering}
\begin{tabular}{|c|c|c|c|}
  \hline
  Geometry       & X position (cm) & Y position (cm) & Z position (cm) \\
  \hline  \hline
  Toy problem I  & -11, 11         & -22, 22         & -93, 93         \\
  \hline
  Toy problem II  & -22, 22         & -44, 44         & -186, 186       \\
  \hline
\end{tabular}
\caption{Toy problems dimensions}
\label{tab:geom_dim}
\end{centering}
\end{table}

All problems were run with a 1 GeV proton source in the z direction and 
with 1E6 particles. 

Six geometries were created, three for each toy problem. 
The first geometry was built as  described above, the second geometry 
was split into 8 equal cells and the last one was split 64 equal cells. 
Splitting toy problem 2 into 8 equal cells required that the materials 
be mixed in a 3:1 ratio


\subsection{Workflow to Photon Emissions}

In oder to generate the spectrum file, the following steps were carried out:
\begin{itemize}
\item MCNP run 
\item Generation of 
\end{itemize}
The following runs were performed:

\begin{itemize}
  \item rnucs on cells 
  \item 1x1x1 mesh
  \item 2x2x2 mesh
  \item 4x4x4 mesh 
  \item 2x2x2 split
  \item 4x4x4 split
\end{itemize}

Each mesh covered the entire geometry and was uniformly distributed. 
Splitting toy problem II into 8 equal parts required the mixed material to be created. 
The mixed material was 3:1 ratio of steel and mercury. 

\subsection{Activation}

An activation calculation was done for each run for a set of irradiation history. 





