\section{Nuclide concetration equations}
This section is primarily to write out equations that might be
important to include when talking about the mesh radionuclide
workflow. 
%%
%%
The rate of change of nuclear concentration is given by
\begin{equation}\label{eq:nuclide_conc_rc}
    \frac{dn_{i}(t)}{dt} = \sum_{j} n_{j}p_{j \rightarrow i, total}
    - \sum_{j} n_{i}p_{i \rightarrow j, total}
\end{equation}
%%
%%
For multiple nuclides this becomes a system of differential equations and
can be represented in the following notation
\begin{equation}\label{eq:nuclide_conc_rc_vec}
  \frac{d\vec{N}}{dt} =\boldsymbol{A}  \vec{N}(t)
\end{equation}
%%
%%
A solution to the system of differential equations is given by equation \ref{eq:rate_change_sol}
\begin{equation}\label{eq:rate_change_sol}
  \vec{N} =\vec{N}_{o} e^{\boldsymbol{A}t}
\end{equation}
A way to solve Equation \ref{eq:rate_change_sol} is to break the transmutation network
into a collection of linear chains so that each isotope has one production term and
one destruction term making the $\boldsymbol{A}$ matrix bidiagonal allowing a solution
in the form of the Bateman Equation.
\begin{equation}\label{eq:bateman}
  N_{i}(t) = \sum_{j=1}^{i} N_{j}(0)
  \Bigg[ \Bigg( \prod_{k=j}^{i-1} P_{k} \Bigg)
  \Bigg(\sum_{k=j}^{i}\frac{e^{-d_{k}t}}{\prod_{l=j,\neq k}^{i}(d_{l} -d_{k})}
  \Bigg)
  \Bigg]
\end{equation}
where $d$ represents the destruction term and $P$ the production ter term
%%
%%
When there is not nuclear data availble, like is the case for high energy interactions,
another term is introduced in Equation \ref{eq:nuclide_conc_rc}. This term is a
production rate constant.
\begin{equation}\label{eq:nuclide_conc_rc_he}
    \frac{dn_{i}(t)}{dt} = \sum_{j} n_{j}p_{j \rightarrow i, total}
    - \sum_{j} n_{i}p_{i \rightarrow j, total} + Y_{i}
\end{equation}
%%
%%
The system of differential

