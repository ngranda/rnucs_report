\section{Implementation}
The previous SDR workflow has a fidelity limited to individual
geomtric volumes. The implementation discussed in this work
supports spatial dicretization via superimposed Cartesian meshes. 
This implemetation required the 
development of a collection of scripts and patches that 
ultimately form the SDR workflow for high energy systems 
with spatial discretization. The workflow can be seen in Figure
\ref{mesh_rnucs}

\begin{figure}[ht]
\begin{centering}
\includegraphics[scale=0.4]{../figs/mesh_rnucs_r2s.png}
\caption{Mesh RNUCS Workflow}
\label{mesh_rnucs}
\end{centering}
\end{figure}


\subsection{MCNP Patch}
The first step in the SDR  workflow is the neutron transport step.
Several changes in the MCNP source code were necessary in order to 
collect radionuclide information in a superimposed mesh.
These MCNP changes were recorded in a patch file. This file should be applied 
directly to MCNP6.1 after a DAGMC patch has been applied.
This patch DAG-MCNP6.1 will output a file with the name $r\_mesh$ which stores 
radionuclide information per voxel. 

\subsection{Activation Script}
A python script was created to collect mesh rnucs information from the $r\_mesh$
file, flux information from a meshtal file, and material information from a 
material laden CAD geometry. This collected information is then written out 
in the correct format and in the right files that can be read by the activation 
software CINDER90. 

The original perl activation script is used to read a keyword $mesh$ and call the 
python script mentioned above, write an input file for each region/voxel and run 
the CINDER90 and TABCODE software. 

\newpage
